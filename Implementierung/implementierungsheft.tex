\documentclass[parskip=full]{scrartcl}

\usepackage[utf8]{inputenc}
\usepackage[T1]{fontenc}
\usepackage[german]{babel}
\usepackage{hyperref}
\usepackage{mathptmx} 
\usepackage[scaled]{helvet} %
\usepackage{graphicx}
\hypersetup{
pdftitle={Implementierungsheft},
bookmarks=true,
}

\makeatletter
\setlength{\@fptop}{0pt}
\makeatother

\usepackage{csquotes} 

{
\titlehead{\centering\includegraphics[width=10cm]{Logo.png}}
\title{\fontsize{40}{48} \selectfont \textsc{Implementierungsheft}\\
{\fontsize{18}{18} \selectfont Multimediatool zum Testen von Videoencodern}}}
\author {Jan Benedikt Schwarz, Johannes Werner, Noel Schuhmacher, Carina Weber,\\
Sascha Rapp, Simon Grafenhorst}

\begin{document}
\maketitle
\thispagestyle{empty}
\newpage
\tableofcontents
\newpage
\section{Entwurfsänderungen}
\subsection{Allgemeines}
\begin{itemize}
\item Korrigierte Rechtschreibfehler in den Klassen-/Methodennamen werden nicht dokumentiert. 
\item Allen Attributen wird ein Unterstrich zur bessern Lesbarkeit des Codes angehaengt.
\item Gui-Klassen werden folgende Methoden bei Bedarf hinzugefuegt:
\begin{verbatim}
(private) void createUi()
(private) void connectActions()
\end{verbatim}
Diese Methoden dienen allein der Lesbarkeit des Codes.

\item Aenderungen an den folgenden Methodenspezifizieren werden nicht dokumentiert:
\begin{verbatim}
const
noexcept
\end{verbatim}

\end{itemize}
\subsection{Filterconfigurationboxes}
\subsubsection{Basisklasse: FilterConfigurationBox}
\begin{verbatim}(public) void setFilterTab(FilterTab& filterTab)
(public) void setFilterIndex(std::size_t index)
(private) std::size_t index
(private) FilterTab* filterTab
\end{verbatim}
Dass der UndoCommand zum Rueckgaengig machen der Aenderung korrekt erstellt werden kann.
\begin{verbatim}
(public) virtaul void updateUi()
\end{verbatim}
Um der FilterConfigurationBox bescheid sagen zu koennen, dass der Filter sich ausserhalb der Box geaendert hat.
\begin{verbatim}
(protected) void updatePreview()
\end{verbatim}
Wird von der erbenden Klasse aufgerufen, um die Vorschau entsprechend den Aenderungen des Benutzers anzupassen.
\begin{verbatim}
(protected) void updateTempFilter();
\end{verbatim}
Wird von der erbenden Klasse aufgerufen, um den temporaeren Filter auf den Stand des originalen Filters zu bringen.
\begin{verbatim}
(private slots) applyFilter()
(private slots) resetFilter()
\end{verbatim}
Slots fuer die entsprechenden Buttons.
\begin{verbatim}
(private) static QImage& getDefaultImage()
(private) static std::unique_ptr<QImage> defaultImage
\end{verbatim}
Stellt das Standardbild zur Vorschau zur Verfuegung.
\begin{verbatim}
(protected) std::unique_ptr<Model::Filter> tempFilter
\end{verbatim}
Temporaeres Filterobjekt, auf welched die erbende Klasse die Aenderungen des Benutzers anwendet.
\begin{verbatim}
(protected) QScrollArea* filterOptionsArea
(private) std::unique_ptr<QImage>
(private) FrameView* filterPreview
(private) QLabel* label_filter
(private) QPushButton* button_apply
(private) QPushButton* button_reset
\end{verbatim}
Diese Attribute sind zur Steuerung der Gui.



\end{document}