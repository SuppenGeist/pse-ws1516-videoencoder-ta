%%% Template originaly created by Karol Kozioł (mail@karol-koziol.net) and modified for ShareLaTeX use
\documentclass[parskip=full]{scrartcl}

\usepackage[utf8]{inputenc} % use utf8 file encoding for TeX sources
\usepackage[T1]{fontenc} % avoid garbled Unicode text in pdf
\usepackage[german]{babel} % german hyphenation, quotes, etc
\usepackage{hyperref} % detailed hyperlink/pdf configuration
\hypersetup{ % ‘texdoc hyperref‘ for options
pdftitle={Protokoll},
bookmarks=true,
}
\usepackage{csquotes} % provides \enquote{} macro for "quotes"
\usepackage{graphicx}
\usepackage{xcolor}

\usepackage{tgtermes}
\usepackage{amsmath,amssymb,amsthm,textcomp}
\usepackage{enumerate}
\usepackage{multicol}
\usepackage{tikz}

\usepackage{geometry}
\geometry{total={210mm,297mm},
left=25mm,right=25mm,%
bindingoffset=0mm, top=20mm,bottom=20mm}


\linespread{1.3}

\newcommand{\linia}{\rule{\linewidth}{0.5pt}}

% custom theorems if needed
\newtheoremstyle{mytheor}
    {1ex}{1ex}{\normalfont}{0pt}{\scshape}{.}{1ex}
    {{\thmname{#1 }}{\thmnumber{#2}}{\thmnote{ (#3)}}}

\theoremstyle{mytheor}
\newtheorem{defi}{Definition}

% my own titles
\makeatletter
\renewcommand{\maketitle}{
\begin{center}
\vspace{2ex}
{\huge \textsc{\@title}}
\vspace{1ex}
\\
\linia\\
\@author \hfill \@date
\vspace{4ex}
\end{center}
}
\makeatother
%%%

% custom footers and headers
\usepackage{fancyhdr,lastpage}
\pagestyle{fancy}
\lhead{}
\chead{}
\rhead{}
\cfoot{}
\rfoot{\thepage}
\renewcommand{\headrulewidth}{0pt}
\renewcommand{\footrulewidth}{0pt}
%

%%%----------%%%----------%%%----------%%%----------%%%

\begin{document}

\title{Protokoll \today}
\date{\today}

\maketitle
\section{Überlegungen}
\begin{itemize}
\item Maximale Video-Auflösung definieren
\item Sich bewegende Artefakte
\item Vorschau-Frames auswählen
\item Import Video-Codecs definieren
\item Was passiert, wenn man versucht komplett andere Videos zu vergleichen?
\end{itemize}

\section{Tools}
Visual Paradigm für UML

\section{Libraries}
\begin{itemize}
\item GStreamer, was bietet das?
\item Makroblöcke
\end{itemize}

\section{Pflichtenheft}
\begin{itemize}
\item Namen auf Titelseite
\item Logo
\item Nur ein Referenzvideo
\item Filter selber schreiben
\item Sonst ffmpeg, libav?
\item 1.1.2: Mehrere encodierte Videos laden, auch Referenzvideo?
\item Wir haben eine GUI -> Musskriterium
\item Analyseergebnisse speichern? Berechnung könnte länger dauern... -> Wunschkriterien
\item Pentium 4?
\item Ressourcenverbrauch definieren, Festplattenspeicher könnte auch mehr sein, 5gb
\item Monitorauflösung... Farbmonitor?
\item Software: Bibliotheken
\item Musskriterien funktionale Anforderungen: Frame vor, Frame zurück
\item Kompatible Dateiformate auflisten
\item Filter: Helligkeit? Extra Filter wäre evtl besser
\item Beim Speichern der Videos: Dateiformat dasselbe wie bei input Video. Evtl nicht Rohdatenvideo filtern, Ausgabeformat von diesen? Eher genau definieren, welches Ausgabeformat
\item Bei jedem Video wird ein Diagramm angezeigt, ohne Auswahl
\item Rechtschreib- und Tippfehler
\item Außer PSNR noch was?
\item Produktdaten: Testvideos
\item Responsive...?
\item Wie lange dauert...? Programmstart, Filter anwenden, usw
\item GUI nur Deutsch, nur Englisch?
\item Was sieht man in der GUI?
\item Filter und Artefakte, wo sieht man die Vorschau?
\item Volume Button? Audio?
\item Audio? Wird mit abgespielt?
\item Load Video Button? Erklären
\item Die letzten Videos sieht man nicht mehr?
\item Wo werden die letzten Videos angezeigt?
\item Letzte Videos werden global oder im Projekt gespeichert? -> Produktdaten
\item Analyse-Ansicht: Wo können Videos abgespielt werden?
\item Wie sollen die Videos verglichen werden? Add comparison...
\item Konzept evtl nochmal überdenken? Vergleich, abspielen, ...
\item Makroblöcke ändern sich mit der Zeit
\item GUI Bilder fehlen, add comparison
\item Umständliches GUI Konzept...
\item Nutzerfreundlichkeit
\item Systemmodelle, Texte
\item Ausführlichere Testszenarien: Was macht der Nutzer, was passiert (Nutzer klickt auf Button, es öffnet sich etwas...)
\item Fehlerszenarien
\item Seitenanzahl: i.O. ca 30
\end{itemize}

\section{Abgabe}
Nächsten Samstag, wie immer, und Montag ausgedruckt in Farbe mit Hefter
Präsentation: 15 Minuten, Zeit einhalten
\end{document}