%%% Template originaly created by Karol Kozioł (mail@karol-koziol.net) and modified for ShareLaTeX use
\documentclass[parskip=full]{scrartcl}

\usepackage[utf8]{inputenc} % use utf8 file encoding for TeX sources
\usepackage[T1]{fontenc} % avoid garbled Unicode text in pdf
\usepackage[german]{babel} % german hyphenation, quotes, etc
\usepackage{hyperref} % detailed hyperlink/pdf configuration
\hypersetup{ % ‘texdoc hyperref‘ for options
pdftitle={Vorlage},
bookmarks=true,
}
\usepackage{csquotes} % provides \enquote{} macro for "quotes"
\usepackage{graphicx}
\usepackage{xcolor}

\usepackage{tgtermes}
\usepackage{amsmath,amssymb,amsthm,textcomp}
\usepackage{enumerate}
\usepackage{multicol}
\usepackage{tikz}

\usepackage{geometry}
\geometry{total={210mm,297mm},
left=25mm,right=25mm,%
bindingoffset=0mm, top=20mm,bottom=20mm}


\linespread{1.3}

\newcommand{\linia}{\rule{\linewidth}{0.5pt}}

% custom theorems if needed
\newtheoremstyle{mytheor}
    {1ex}{1ex}{\normalfont}{0pt}{\scshape}{.}{1ex}
    {{\thmname{#1 }}{\thmnumber{#2}}{\thmnote{ (#3)}}}

\theoremstyle{mytheor}
\newtheorem{defi}{Definition}

% my own titles
\makeatletter
\renewcommand{\maketitle}{
\begin{center}
\vspace{2ex}
{\huge \textsc{\@title}}
\vspace{1ex}
\\
\linia\\
\@author \hfill \@date
\vspace{4ex}
\end{center}
}
\makeatother
%%%

% custom footers and headers
\usepackage{fancyhdr,lastpage}
\pagestyle{fancy}
\lhead{}
\chead{}
\rhead{}
\cfoot{}
\rfoot{\thepage}
\renewcommand{\headrulewidth}{0pt}
\renewcommand{\footrulewidth}{0pt}
%

%%%----------%%%----------%%%----------%%%----------%%%

\begin{document}

\title{Protokoll \today}
\date{\today}

\maketitle

\section{Überlegungen zum Pflichtenheft}

Erstmal keine Klassendiagramme oder Modelle ins Pflichtenheft bis auf GUI. 


\section{Musskriterien}
Das Programm muss:

\begin{itemize}
\item Zwei Videos abspielen
\item Frame by Frame spulen unterstützen
\item Videofilter und Artefakte einfügen
\item Die Videos automatisch bewerten können via PSNR
\item Makroblöcke anzeigen
\item Encoder aufrufen und diesem ein (vorher modifiziertes) Video übergeben
\end{itemize}

\section{Wunschkriterien}
Das Programm soll:
\begin{itemize}
\item Fast Forward Funktion unterstützen
\item Eigene Bewertung in Datenbank abspeichern
\item Mehrere Encoder gleichzeitig vergleichen
\item Hervorheben von Unterschieden
\end{itemize}

\end{document}
