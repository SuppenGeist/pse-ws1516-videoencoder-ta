\documentclass{scrartcl}

\usepackage[german]{babel}
\usepackage{graphicx}
\usepackage[utf8]{inputenc}
\usepackage[T1]{fontenc}
\usepackage[pdfborderstyle={/S/U/W 1}]{hyperref}
\usepackage[scaled]{helvet}
\usepackage{geometry}

\geometry{a4paper,left=30mm,right=30mm, top=3cm, bottom=3cm}

{
\titlehead{\centering\includegraphics[width=10cm]{Logo.png}}
\title{\fontsize{40}{48} \selectfont \textsc{Qualitätssicherung}\\
{\fontsize{18}{18} \selectfont Multimediatool zum Testen von Videoencodern}}}
\author {Johannes Werner, Noel Schuhmacher, Sascha Rapp,\\ Simon Grafenhorst,
Carina Weber, Jan Benedikt Schwarz}
 
\begin{document} {
\maketitle
\thispagestyle{empty}
\pagestyle{empty}
\newpage
\setcounter{page}{0}
\pdfbookmark{\contentsname}{toc}
\tableofcontents
\clearpage
\pagestyle{plain}
\newpage
\section{Einleitung}
In der Qualitätssicherungsphase geht es darum die Qualität des Projectes  durch intensives testen zu prüfen. Dazu werden sowohl Unit-Tests als auch Manuelles Testen verschiedener Szenarien sowie Testen durch dritte Pesrsonen hinzugezogen.
\paragraph{2 Bugs} In diesem Kapitel werden gefundene Bugs und deren Reparatur dokumentiert.
\paragraph{3 Testüberdeckung} Dieses Kapitel gibt einen kurzen Einblick in die Testüberdeckung des Codes.
\newpage
\section{Bugs}
\subsection{PSNRDifferenceCalculator bug}
\textbf{Symptome:}\\
\textbf{Ursache:} Fehlende if Bedingung ob es sich um ähnliche Videos handelt.\\
\textbf{Behebung:} Hinzufügen einer Überprüfung ob die Videos die selbe Auflösung haben.
\subsection{move Filter bug}
\textbf{Symptome:} Error beim verschieben der Filter in der Filterliste.\\
\textbf{Ursache:} Falsche newposition in den if cases der moveFilter Methode.\\
\textbf{Behebung:} Ändere filters\_size() zu filters\_size()-1.
\subsection{VideoLoader bug}
\textbf{Symptome:}\\
\textbf{Ursache:}\\
\textbf{Behebung:}
\subsection{VideoSaver bug}
\textbf{Symptome:}\\
\textbf{Ursache:}\\
\textbf{Behebung:}
\subsection{Bug in mementos}
\textbf{Symptome:} Comment wird nicht mit abgespeichert.\\
\textbf{Ursache:} Comment wird nicht in den String, der zum Speichern benutzt wird geschrieben.\\
\textbf{Behebung:}Comment wird mit memento->getComment() an den String angehängt.
\subsection{Loading bug}
\textbf{Symptome:}\\
\textbf{Ursache:}\\
\textbf{Behebung:}
\subsection{FPS bug}
\textbf{Symptome:}\\
\textbf{Ursache:}\\
\textbf{Behebung:}
\subsection{Filter bug}
\textbf{Symptome:}\\
\textbf{Ursache:}\\
\textbf{Behebung:}
\subsection{Broken video file bug}
\textbf{Symptome:} Unzulässige Dateien können als Videos geladen werden.\\
\textbf{Ursache:} Fehlende Überprüfung ob es sic bei der Datei um ein Video handelt.\\
\textbf{Behebeung:} Hinzufügen einer Überprüfung ob die Datei Frames enthält.
\subsection{Project loading bug}
\textbf{Symptome:} Nach Laden des Projects undo und redo nicht ausführbar\\
\textbf{Ursache:} Der UndoStack wird an der Falschen Stelle gelehrt.\\
\textbf{Behebung:} Verschiebung von UndoRedo::UndoStack::getUndoStack().clear(); vom Anfang der restore Methode im MainWindow an das Ende der Methode.
\subsection{Video saving bug}
\textbf{Symptome:}\\
\textbf{Ursache:}\\
\textbf{Behebung:}
\subsection{Project loading bug 2}
\textbf{Symptome:}\\
\textbf{Ursache:}\\
\textbf{Behebung:}\\
\newpage
\section{Testüberdeckung}
Durch das hinzufügen weiterer Unit tests und automatischer GUI tests konnten wir die TestAbdeckung von den 23\% aus der Implementierungsphase auf ca 73\% erweitern.
\end{document}