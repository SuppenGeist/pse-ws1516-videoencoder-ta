\documentclass{scrartcl}

\usepackage[german]{babel}
\usepackage{graphicx}
\usepackage[utf8]{inputenc}
\usepackage[T1]{fontenc}
\usepackage[pdfborderstyle={/S/U/W 1}]{hyperref}
\usepackage[scaled]{helvet}
\usepackage{geometry}

\geometry{a4paper,left=30mm,right=30mm, top=3cm, bottom=3cm}

{
\titlehead{\centering\includegraphics[width=10cm]{Logo.png}}
\title{\fontsize{40}{48} \selectfont \textsc{Qualitätssicherung}\\
{\fontsize{18}{18} \selectfont Multimediatool zum Testen von Videoencodern}}}
\author {Johannes Werner, Noel Schuhmacher, Sascha Rapp,\\ Simon Grafenhorst,
Carina Weber, Jan Benedikt Schwarz}
 
\begin{document} {
\maketitle
\thispagestyle{empty}
\pagestyle{empty}
\newpage
\setcounter{page}{0}
\pdfbookmark{\contentsname}{toc}
\tableofcontents
\clearpage
\pagestyle{plain}
\newpage
\section{Einleitung}
In der Qualitätssicherungsphase geht es darum die Qualität des Projectes  durch intensives testen zu prüfen. Dazu werden sowohl Unit-Tests als auch Manuelles Testen verschiedener Szenarien sowie Testen durch dritte Personen hinzugezogen.\\
In diesem Dokument werden die gefundenen Bugs mit ihren Symptomen, Ursachen und Behebungen festgehalten. Des weiteren wird eine Übersicht zur Testabdeckung gegeben und es werden zusätzliche für die Qualitätssicherung erstellte Szenarien beschrieben.


\newpage
\section{Bugs}
In diesem Kapitel werden gefundene Bugs und deren Reparatur dokumentiert.
\subsection{PSNRDifferenceCalculator bug}
\textbf{Symptome:}Falsche Anzeigen bei den PSNR Graphen.\\
\textbf{Ursache:} Fehlende if Bedingung ob es sich um ähnliche Videos handelt.\\
\textbf{Behebung:} Hinzufügen einer Überprüfung ob die Videos die selbe Auflösung haben.
\subsection{move Filter bug}
\textbf{Symptome:} Fehlermeldung beim verschieben der Filter in der Filterliste.\\
\textbf{Ursache:} Falsche newposition in den if cases der moveFilter Methode.\\
\textbf{Behebung:} Ändere filters\_size() zu filters\_size()-1.
\subsection{VideoLoader bug}
\textbf{Symptome:}
Encodierte Videos werden nicht korrekt geladen.\\
\textbf{Ursache:}
Geladene Videos wurden nicht zu RGB24 umgewandelt.\\
\textbf{Behebung:}
Umwandeln von yuv420 zu rgb24.
\subsection{VideoSaver bug}
\textbf{Symptome:}
Alle abgespeicherten RGB Histogramvideos haben die selbe Farbe.\\
\textbf{Ursache:}
Falsches setzen der Farbe, in denen die Histogramme gerendert werden.\\
\textbf{Behebung:}
Korrektes setzten der Farben.
\subsection{Bug in mementos}
\textbf{Symptome:} Comment wird nicht mit abgespeichert.\\
\textbf{Ursache:} Comment wird nicht in den String, der zum Speichern benutzt wird geschrieben.\\
\textbf{Behebung:}Comment wird mit memento->getComment() an den String angehängt.
\subsection{Loading bug}
\textbf{Symptome:}
Im Analysistab werden beim Laden eines Projektes manche Videos nicht geladen (inkonsistent).\\
\textbf{Ursache:}
Entscheidung, ob ein Konverter auf das vollständig geladene Video wartet hängt vom Timing der threads ab.\\
\textbf{Behebung:}
Timingsunabhängig prüfen, ob gewartet werden muss.
\subsection{FPS bug}
\textbf{Symptome:}
Gespeicherte Videos der RGB Histogramme haben nicht die korrekte FPS.\\
\textbf{Ursache:}
Fps wird nicht korrekt gesetzt.\\
\textbf{Behebung:}
Setzen der Fps.
\subsection{Filter bug}
\textbf{Symptome:}
Filtertab ist beim Hinzufügen eines Filters zur Filterpreview gewechselt, obwohl kein Video geladen war.\\
\textbf{Ursache:}
Es wird nicht geprüft, ob ein Video geladen ist.\\
\textbf{Behebung:}
Prüfen, ob ein Video geladen ist.
\subsection{Broken video file bug}
\textbf{Symptome:} Unzulässige Dateien können als Videos geladen werden.\\
\textbf{Ursache:} Fehlende Überprüfung ob es sich bei der Datei um ein Video handelt.\\
\textbf{Behebeung:} Hinzufügen einer Überprüfung ob die Datei Frames enthält.
\subsection{Project loading bug}
\textbf{Symptome:} Beim Laden eines Projektes stürzt das Programm im Analysistab ab.\\
\textbf{Ursache:} UndoStack wird nach dem Laden des Projektes geleert, somit auch die geladenen Videos.\\
\textbf{Behebung:} UndoStack wird vor dem Laden des Video geleert.
\subsection{Video saving bug}
\textbf{Symptome:}
Gespeicherte Videos können nicht von einem externen Player abgespielt werden.\\
\textbf{Ursache:}
Metadaten zum Video werden nicht gespeichert.\\
\textbf{Behebung:}
Speichern von Metadaten.
\subsection{Write comment bug}
\textbf{Symptome:} nach benutzen von undo, gefolgt von redo wird das letzte Wort in der commentbox abgeschnitten\\
\textbf{Ursache:} Schlechte if Bedingung für das erstellen eines undo Command.\\
\textbf{Behebung:} Verwendung von merge im write command und entfernen der if Bedingung.
\subsection{Project loading bug 2}
\textbf{Symptome:}
Beim Laden eines Projekts, wenn bereits ein Video geladen ist, stürzt das Programm ab.\\
\textbf{Ursache:}
Der Undostack wird gelöscht, der Player aber nicht gestoppt. Der Player hat versucht ein gelöschtes Video abzuspielen.\\
\textbf{Behebung:}
Stoppen des Players vor dem Löschen des UndoStacks.\\
\subsection{Filterlist undo bug}
\textbf{Symptome:}
Benutzen von undo nach hinzufügen/entfernen/reseten bei der Filterliste.\\
\textbf{Ursache:}
Alter Zustand wird mit restore(memento) wiederhergestellt.\\
\textbf{Behebung:}
Statt restore(memento) wird eine Handlung simuliert die das Ganze rückgängig macht.
\newpage
\section{Testabdeckung}
Die Testabdeckung von den 23\% aus der Implementierungsphase konnten wir durch das hinzufügen weiterer Unit tests und automatischer GUI tests auf ca. 74\% erweitern.\\
Durch Manuelle Tests konnten wir die Abdeckung nochmal auf bis zu 84\% erweitern. Zu diesen gehören das durchgehen der Szenarien die im Pflichtenheft beschrieben sind. Des weiteren haben wir auch noch das speichern und laden des Projektes, der Results und Filterconfigurationen sowie diverser Videos in verschiedenen Formaten manuell getestet.
\subsection{Testabdeckung der einzelnen Szenarien}
\subsubsection{Szenario 1}
\textit{Szenario 1 aus dem Pflichtenheft.}\\
Dieses Szenario ergab eine Testabdeckung von 49\%.
\subsubsection{Szenario 2}
\textit{Szenario 2 aus dem Pflichtenheft.}\\
Dieses Szenario ergab eine Testabdeckung von 62\%.
\subsubsection{Szenario 3}
\textit{Szenario 3 aus dem Pflichtenheft.}\\
Dieses Szenario ergab eine Testabdeckung von 48\%.
\subsubsection{Szenario 4}
\textit{CrazyMonkeyTest siehe Zusätzliche Szenarien und Testfälle.}\\
Dieses Szenario ergab eine Testabdeckung von 31\%.
\newpage
\section{Zusätzliche Szenarien und Testfälle}
Zum testen der GUI haben wir zusätzliche zu den aus Szenarien aus dem Pflichtenheft folgende automatisch ausgeführte Mini-Szenarien erstellt.
\subsection{CrazyMonkeyTest}
Hintereinanderausführung 'zufälliger' Aktionen.\\
\\
\textbf{Ablauf:}\\
\\
Starte im \textit{Filtertab}\\
Klicke Button: Apply to video\\
Klicke Button: ''Filter up''\\
Klicke Button: ''Apply to video''\\
Klicke Button: ''Filter down''\\
Klicke Button: ''Save configuration''\\
Klicke Button: ''Add Edge filter''\\
Wechsle zu \textit{Analysistab}\\
Klicke Button: ''PSNR''\\
Klicke Button: ''Bitrate''\\
Klicke Button: ''Red histogram''\\
Lade Video\\
Klicke Button: ''PSNR''\\
Klicke Button: ''Bitrate''\\
Klicke Button: ''Red histogram''\\
Wechsle zu \textit{FilterTab}\\
Klicke Button: ''Remove filter''\\
Klicke Button: ''Save configuration''\\
Klicke Button: ''Add Contrast filter''\\
Lade Video\\
Klicke Button: ''Add Vintage filter''\\
Klicke Button: ''Apply to video''\\
Klicke Button: ''Redo''\\
Klicke Button: ''Redo''\\
Klicke Button: ''Redo''\\
Klicke Button: ''Add Vintage filter''\\
Klicke Button: ''Redo''\\
Klicke Button: ''Undo''\\
Wechsle \textit{Analysistab}\\
Klicke Button: ''New''\\
Klicke Button: ''Add video''\\
\subsection{AddFilter ohne Video}
Filter hinzufügen ohne zuvor ein Video geladen zu haben.

\end{document}
