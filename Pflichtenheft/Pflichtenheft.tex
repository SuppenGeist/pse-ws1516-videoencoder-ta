\documentclass[parskip=full]{scrartcl}

\usepackage[utf8]{inputenc} % use utf8 file encoding for TeX sources
\usepackage[T1]{fontenc} % avoid garbled Unicode text in pdf
\usepackage[german]{babel} % german hyphenation, quotes, etc
\usepackage{hyperref} % detailed hyperlink/pdf configuration
\usepackage{mathptmx} %Schriftart Times
\usepackage[scaled]{helvet} %
\hypersetup{ % ‘texdoc hyperref‘ for options
pdftitle={Pflichtenheft},
bookmarks=true,
}
\usepackage{csquotes} % provides \enquote{} macro for "quotes"
%TitlePage
{\title{\fontsize{40}{48} \selectfont \textsc{Pflichtenheft}\\
{\fontsize{18}{18} \selectfont Multimediatool zum Testen von Videoencodern}}}
{\author{Carina, Ben, Johannes, Noel, Sascha, Simon}}

\begin{document}
\maketitle
\thispagestyle{empty}
\newpage
\tableofcontents
\newpage
\section{Zielbestimmung}
Das Programm soll ermöglichen, verschiedene Videocodecs auf ihre Qualität zu untersuchen.
\subsection{Musskriterien}

\subsection{Wunschkriterien}

\section{Produkteinsatz}

\subsection{Anwendungsbereiche}

\subsection{Zielgruppe}

\subsection{Betriebsbedingungen}

\section{Produktumgebung}

\subsection{Hardware}
Für die Software wird ein 64 Bit Betriebssystem benötigt, welches einen 64 Bit Prozessor voraussetzt.

\subsection{Software}
\begin{itemize}
\item 64Bit Linux
\item Qt Bibliothek
\end{itemize}

\section{Funktionale Anforderungen}

\section{Produktdaten}
\subsection{Abgespeicherte Daten}
Die drei zuletzt verwendeten Videos werden abgespeichert, sodass diese über ein Dropdown-Auswahlmenü schnell ausgewählt werden können. Zu dem selben Zweck werden die drei zuletzt verwendeten Videoencoder abgespeichert.
\subsection{Vergleichsdaten}
\begin{itemize}
\item Ausgewähler Encoder
\item Ausgewähltes Video
\item Angewendete Filter und Artefakte
\item Generierte PSNR Daten
\end{itemize}
\section{Nichtfunktionale Anforderungen}

\section{Benutzungsoberflächer}

\section{Qualitätsbestimmungen}

\section{Globale Testfälle und Szenarien}


\end{document}