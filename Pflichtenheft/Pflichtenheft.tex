\documentclass[parskip=full]{scrartcl}

\usepackage[utf8]{inputenc} % use utf8 file encoding for TeX sources
\usepackage[T1]{fontenc} % avoid garbled Unicode text in pdf
\usepackage[german]{babel} % german hyphenation, quotes, etc
\usepackage{hyperref} % detailed hyperlink/pdf configuration
\usepackage{mathptmx} %Schriftart Times
\usepackage[scaled]{helvet} %
\usepackage{graphicx}
\hypersetup{ % ‘texdoc hyperref‘ for options
pdftitle={Pflichtenheft},
bookmarks=true,
}
\usepackage{csquotes} % provides \enquote{} macro for "quotes"
%TitlePage
{\title{\fontsize{40}{48} \selectfont \textsc{Pflichtenheft-Entwurf}\\
{\fontsize{18}{18} \selectfont Multimediatool zum Testen von Videoencodern}}}
{\author{Carina Weber, Jan Benedikt Schwarz, Johannes Werner, Noel Schuhmacher, Sascha Rapp, Simon Grafenhorst}}

\begin{document}
\maketitle
\thispagestyle{empty}
\newpage
\tableofcontents
\newpage
\section*{Einleitung}
Vive (lang: Video veritatem) ist ein Programm zum Testen verschiedener Videoencoder. Man hat die Möglichkeit ein Video (mit Filtern) zu bearbeiten welches dann von einem externen Encoder encodiert wird. Dieses encodierte Video kann dann wieder in Vive geladen werden, wo so komfortabel mit graphischen Visualisierungen entschieden werden kann, wie gut der Encoder das Video encodiert hat.
\section{Zielbestimmung}
Vive ist ein Multimedia-Framework zum Vergleichen und zur Evaluation von Videoencodern.
\subsection{Musskriterien}
\subsubsection{Video Auswählen}
Der Benutzer muss können:
\begin{itemize}
\item Video laden
\item Aus einer Liste kürzlich geladener Videos auswählen
\item Aus einer Liste von Filtern mehrere Filter auswählen und diese auf das gesamte Video anwenden
\item Aus einer Liste von Artefakten ein Artefakt auswählen und dieses in das gesamte Video einfügen
\item Anzeige einer Vorschau nach Auswahl der Filter und Artefakte
\item Video abspeichern
\end{itemize}
\subsubsection{Veränder des Videos}
\begin{itemize}
\item Video mithilfe von Filtern und Artefakten verändern
\item Mindestens drei Filter gleichzeitig anwenden
\item Zusätzlich zu den Filtern noch ein Artefakt auswählen
\item Das veränderte Video abspeichern
\end{itemize}
\subsubsection{Bewerten des Encoders}
\begin{itemize}
\item Rohvideo und das encodierte Video synchron nebeneinander anschauen
\item Die im vorigen Punkt genannten Videos in einstellbarer Geschwindigkeit abspielen
\item Berechnete Bewertungskriterien für den Encoder anzeigen
\item Unterschiede zwischen rohem und encodiertem Video in einer Differenzanzeige anzeigen lassen
\item Encoderspezifische Eigenschaften wie Makroblöcke und Bitrate anzeigen
\item Histogramm anzeigen
\end{itemize}
\subsection{Wunschkriterien}
Der Benutzer soll können:
\begin{itemize}
\item Bewertungen speichern
\item Bewertungen laden und neben der Bewertung des aktuell geladenen Encoders anzeigen
\item Encoder auswählen und über die GUI starten
\item Mehrere Artefakte gleichzeitig anwenden
\end{itemize}
\subsubsection{Projekte}
\begin{itemize}
\item Projekte anlegen
\item Projekte speichern
\end{itemize}
\subsection{Abgrenzungskriterien}
\begin{itemize}
\item Die Software ist ausschließlich in deutscher Sprache verfügbar
\item Die GUI ist fest vorgegeben, d.h. Schriftart,-größe, etc. sind nicht veränderbar
\item Audio wird nicht mit abgespielt oder verglichen
\end{itemize}
\newpage
\section{Produkteinsatz}
Das Produkt dient zum Testen und Vergleichen von Videoencodern. 
Damit sollen sowohl Privatpersonen, als auch Firmen oder Forschungseinrichtungen darin unterstützt werden, den best möglichen Videoencoder für ihre Bedürfnisse zu finden.
\subsection{Anwendungsbereiche}
Vergleich von Videoencodern für verschiedene Bedürfnisse
\subsection{Zielgruppe}
\begin{itemize}
\item Anwender von Videoencodern/ Privatpersonen
\item Hersteller von Videoencodern
\item Videoediteure
\item Graphikdesigner
\item Cutter
\item Videoplattformbetreiber
\item Forschungseinrichtungen
\end{itemize}
\subsection{Betriebsbedingungen}
\begin{itemize}
\item Benutzbar rund um die Uhr
\item Wartungsfrei
\item In privat oder geschäftlichem Einsatz
\end{itemize}
\newpage
\section{Produktumgebung}

\subsection{Hardware}
\begin{itemize}
\item 64 Bit Prozessor
\item Intel: Prozessor der Core-i-Serie oder neuer, AMD: Athlon II, Phenom II oder neuer
\item Mindestens 50 MB freier Speicher
\item Mindestens 4 GB Arbeitsspeicher
\end{itemize}
\subsection{Software}
\begin{itemize}
\item 64Bit Linux
\item Qt Bibliothek Version 5.5.1
\end{itemize}
\newpage
\section{Funktionale Anforderungen}
\subsection{Video auswählen}
Der Benutzer kann mittels eines Dateiauswahldialogs ein Video auszuwählen. Dieses
Video wird später encodiert. Der Benutzer hat die Möglichkeit ein Video aus einer Liste zu wählen, die die letzten 10 encodierten
Videos enthält. Das Programm prüft vor dem Anzeigen der Videos, ob diese noch existieren.
\subsection{Filter auswählen}
Der Benutzer hat die Möglichkeit einen einzigen Filter aus einer Liste vorgefertigter Filter mithilfe von Checkboxen auszuwählen. Vorhandene Filter:
\begin{itemize}
\item Schwarzweiß-Filter
\item Unschärfe-Filter
\item Farbfilter
\end{itemize}
\subsection{Artefakte auswählen}
Der Benutzer hat die Möglichkeit ein einziges Artefakt/Muster aus einer Liste vorgefertigter
Artefakte/Muster auszuwählen. Diese Liste enthält folgende:
\begin{itemize}
\item Gitter-Muster
\end{itemize}
\subsection{Vorschau anzeigen}
Der Benutzer hat die Möglichkeit, wenn er ein Artefakt oder ein Muster ausgewählt hat, sich eine
Vorschau anzeigen zu lassen. Diese Vorschau besteht aus 5 Frames. Diese Frames sind jeweils bei
1/5,2/5,... der gesamten Frameanzahl entnommen. Auf diese Frames wird der Filter bzw. Artefakt
angewendet. Der Benutzer kann durch alle 5 Frames durchschalten. Wurde ein Artefakt und ein Filter
gleichzeitig ausgewählt, wird erst das Artefakt angewendet, dann der Filter und wird die
Vorschau angezeigt.
\subsection{Filter/Artefakte anwenden}
Der Benutzer muss den Filter/Artefakt anwenden, bevor das Video damit encodiert wird. Tut er
das nicht, wird das Originalvideo codiert. Wenn der Nutzer das originale Video verändert, wird eine neue Videodatei mit
dem Filter/Artefakt erzeugt. Werden Filter oder Artefakte geändert und die Änderungen angewandt, wird wieder eine neue
Datei erstellt und die vorherige wieder gelöscht.
\subsection{Vorschau des kompletten Videos}
Hat der Benutzer den Filter/Artefakt angewandt, hat er die Möglichkeit, sich das komplette, neue
Video anzuschauen. Steuerelemente für das Video sind dabei lediglich ein Start, Pause und Restart Buttons.
\subsection{Video Encodieren}
Der Benutzer kann, wenn er ein Video und Encoder ausgewählt hat, das ausgewählte Video mit dem ausgewählten Encoder codieren.
\subsection{Bewertung des Encoders}
\subsubsection{Amschauem des codierten und rohen Videos}
Der Benutzer kann das Rohvideo sowie das codierte anschauen. Dabei sind beide Videos gleichzeitig zu sehen. Steuerelemente sind Start, Pause und Stop sowie eine Timeline. Es gibt nur einen Satz Steuerelemente für beide Videos. D.h. die Videos können nur synchron angeschaut werden.
\subsubsection{Einstellen der Abspielgeschwindigkeit}
Der Benutzer kann die Videos in verschiedenen Geschwindigkeiten abspielen. Zugelassen sind:
\begin{itemize}
\item Frame by Frame
\item 0.25
\item 0.5
\item 0.75
\item 1.0
\item 1.25
\item 1.5
\item 1.75
\item 2.0
\end{itemize}
\subsubsection{Bewertungskriterien anzeigen}
Bewertungskriterien zum Encoder und Video werden angezeigt. Folgende Kriterien/Gegenüberstellungen gibt es:
\begin{itemize}
\item PSNR
\item Dateigröße
\item Encodierungsdauer
\end{itemize}
\subsubsection{Unterschied zwischen Roh- und encodiertem Video anzeigen}
Der Benutzer kann sich die Farbabweichungen jeweils von RGB vom Roh- und encodiertem Video anzeigen lassen.
\subsubsection{Interressante Eigenschaften des encodierten Videos anzeigen}
Der Benutzer kann sich interressante Eigenschaften im encdierten Video anzeigen:
\begin{itemize}
\item Makroblöcke
\end{itemize}
\newpage
\section{Produktdaten}
\subsection{Abgespeicherte Daten}
Die 10 zuletzt verwendeten Videos werden abgespeichert, sodass diese über ein Dropdown-Auswahlmenü schnell ausgewählt werden können. Zu dem selben Zweck werden die 5 zuletzt verwendeten Videoencoder abgespeichert.
\subsection{Vergleichsdaten}
\begin{itemize}
\item Ausgewähler Encoder
\item Ausgewähltes Video
\item Angewendete Filter und Artefakte
\item Generierte PSNR Daten
\end{itemize}
\newpage
\section{Nichtfunktionale Anforderungen}
\begin{itemize}
\item[]\textbf{/L010/ Responsive GUI}\\
Auch bei rechenintensiven Hintergrundaktionen bleibt die GUI responsive.
\item[]\textbf{/L020/ Fehlerrobust}\\
Bei Fehlerhaften Videodateien oder Nutzereingaben soll das Programm nicht abstürzen.
\end{itemize}
\newpage
\section{Benutzungsoberflächer}
\subsection{Anforderungen}
Die Bedienungsoberfläche ist auf Mausbedienung ausgelegt, eine Bedienung ohne Maus muss dennoch möglich sein.
\begin{itemize}
\item DIN 66234, Teil 8 ist zu beachten
\item Die Benutzungsoberfläche wird aus Elementen des Qt Designer aufgebaut
\end{itemize}
\subsection{Beispieldesign}
\subsubsection{Encoderauswahl}
\begin{figure}[htbp] 
\centering
\includegraphics[width=0.4\textwidth]{GUI_Entwurf_1/GUI_1.png}
\caption{Encoderauswahl}
\end{figure}
\begin{itemize}
\item "Öffnen" öffnet einen "Datei öffnen Dialog"
\item Alternativ kann der Dateipfad manuell in das Textfenster eingegeben werden
\item "Zuletzt verwendet" bietet eine Auswahl der kürzlich verwendeten Encoder
\item "Weiter" bestätigt die Auswahl und wechselt zur Videoauswahl
\end{itemize}
\subsubsection{Videoauswahl}
\begin{figure}[htbp] 
\centering
\includegraphics[width=0.5\textwidth]{GUI_Entwurf_1/GUI_2.png}
\caption{Videoauswahl}
\end{figure}
\begin{itemize}
\item "Öffnen" öffnet einen "Datei öffnen Dialog"
\item Alternativ kann der Dateipfad manuell in das Textfenster eingegeben werden
\item "Zuletzt verwendet" bietet eine Auswahl der kürzlich verwendeten Videos
\item "Weiter" bestätigt die Auswahl und wechselt zur Filter- und Musterauswahl
\item "Zurück" wechselt zur Encoderauswahl
\end{itemize}
\subsubsection{Filter/Artefakte}
\begin{figure}[htbp] 
\centering
\includegraphics[width=0.5\textwidth]{GUI_Entwurf_1/GUI_3.png}
\caption{Filter/Artefakte Auswahl}
\end{figure}
\begin{itemize}
\item "Filter" bietet die Möglichkeit einen Filter auszuwählen
\item "Muster/Artefakte" bietet die Möglichkeit ein Muster oder ein Artefakt auszuwählen
\item "Vorschau" berechnet 5 Vorschau Frames
\item "<,>" bieten die Möglichkeit zwischen den Frames zu springen
\item "Video berechnen" wendet die ausgewählten Filter/Muster auf das gesamte Video an
\item "Play, Pause, Restart" bieten die Möglichkeit das berechnete Video abzuspielen
\item "Weiter" bestätigt die Auswahl und wechselt zur Wiedergabe und Auswertung
\item "Zurück" wechselt zur Videoauswahl
\end{itemize}
\subsubsection{Wiedergabe und Auswertung}
\begin{figure}[htbp]
\centering
\includegraphics[width=0.5\textwidth]{GUI_Entwurf_1/GUI_4.png}
\caption{Wiedergabe und Auswertung}
\end{figure}
\begin{itemize}
\item "Play" Startet alle Videos
\item "Pause" Pausiert alle Videos
\item "Restart" Startet alle Videos neu
\item "<" Springt bei allen Videos einen Frame rückwärts
\item ">" Springt bei allen Videos einen Frame vorwärts
\item "x1.0" zeigt die aktuelle Wiedergabegeschwindigkeit und bietet die Möglichkeit diese zu ändern
\item "Differenz" zeigt Differenz oder Makroblöcke in dem Bereich oben rechts
\item "Weitere Informationen" kann Metadaten etc. enthalten
\item "Zurück" Wechselt zur Filter- und Musterauswahl
\item "Encoder ändern" öffnet einen "Datei öffnen Dialog" um den neuen Encoder zu wählen und wechselt dann zu Fenster 3 (alte Auswahl bleibt erhalten)
\end{itemize}
\newpage
\section{Qualitätsbestimmungen}
\begin{tabular}{|c|c|c|c|c|}
\hline & Sehr wichtig & Wichtig & Weniger wichtig & Unwichtig \\
\hline Robustheit & • &  &  & \\ 
\hline Zuverlässigkeit & • &  &  & \\ 
\hline Korrektheit & • &  &  & \\ 
\hline Benutzerfreundlichkeit & • &  &  & \\ 
\hline Effizienz &  & • &  & \\ 
\hline Portierbarkeit &  &  & • & \\ 
\hline Kompatibilität &  &  & • & \\ 
\hline Modifizierbarkeit &  &  & • & \\ 
\hline Sicherheit &  &  &  & • \\ 
\hline 
\end{tabular} 
\newpage
\section{Globale Testfälle und Szenarien}
Folgende Funktionssequenzen sind zu überprüfen:
\subsection{Video Auswählen}
\subsubsection{Video wählen}
"Öffnen" klicken, im Dateisystem das Video wählen
\subsubsection{Video aus Liste wählen}
unter "Zuletzt verwendet" ein Video wählen
\subsubsection{Filter auswählen}
Encoder und Video wählen, einzelne Filter ankreuzen
\subsubsection{Artefakte auswählen}
Encoder und Video wählen, einzelne Artefakte ankreuzen
\subsubsection{Vorschau anzeigen}
Encoder, Video, Filter und Artefakte wählen, "Vorschau drücken", "<", ">" verwenden
\subsubsection{Filter/Artefakte anwenden}
Encoder, Video, Filter und Artefakte wählen, "Video berechnen" drücken
\subsubsection{Vorschau des kompletten Videos}
Encoder, Video, Filter und Artefakte wählen, "Video berechnen" drücken, "Play", "Pause" und "Restart" Button verwenden
\subsection{Video encodieren}
Encoder, Video, Filter und Artefakte anwenden, "weiter" drücken
\subsection{Bewertung des Encoders}
\subsubsection{Anschauen des codierten und rohen Videos}
Encoder, video, Filter und Artefakte wählen, encodieren, unter rechte Seite der Gui betrachten
\subsubsection{Einstellen der Abspielgeschwindigkeit}
Encoder, video, Filter und Artefakte wählen, encodieren, "<", ">" und das Menü rechts von den beiden Buttons verwenden
\subsubsection{Bewertungskriterien anzeigen}
Encoder, video, Filter und Artefakte wählen, encodieren, untere rechte Seite der Gui betrachten
\subsubsection{Unterschied zwischen Roh- und encodiertem Video anzeigen}
Encoder, video, Filter und Artefakte wählen, encodieren, obere rechte Seite der Gui betrachten
\subsubsection{Interressante Eigenschaften des encodierten Videos anzeigen}
Encoder, video, Filter und Artefakte wählen, encodieren, im rechten, mittleren Menü jede Auswahl durchgehen
\section{Glossar}
\subsection*{5-Sterne Bewertungssystem}  
5 Sterne zur Bewertung, wobei 1 Stern die niedrigste Bewertung ist und 5 Sterne die höchste.

\subsection*{Artefakt} Eine Struktur, die über das Video gelegt wird, wie zum Beispiel ein Kreis oder eine Linie.
\subsection*{Benutzer} 
Weibliche oder männliche Person, die das Programm benutzt.
\subsection*{Encoder} 
Ein Programm zum komprimieren von Videodateien.
\subsection*{Filter} 
Ein Algorithmus, der Farbwerte nach einem bestimmten Muster verändert.
\subsection*{Projekt} 
Ein Projekt beinhaltet die Pfade zu den geöffneten Videos sowie die eingestellten Filter und Artefakte.
\subsection*{PSNR-Graph} 
Ein Graph der auf der x-Achse Zeitwerte(Framenummer) enthält und auf der y-Achse den dazugehörigen PSNR-Wert.
\subsection*{PSNR-Wert} 
Peak signal-to-noise ratio ist ein Maß für das Verhältnis zwischen Signalrate und dem überlagerten Rauschsignal.
\subsection*{RGB-Histogramm} 
Ein Graph, der die Farbverteilung eines Videos anzeigt.

\end{document}